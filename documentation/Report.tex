 \documentclass[10pt,a4paper,headinclude=true,twoside]{report}
\usepackage[latin1]{inputenc}
\usepackage[a4paper]{geometry}
\usepackage{a4wide}
\usepackage{amsmath}
\usepackage{amsfonts}
\usepackage{amssymb}
\usepackage{graphicx}
\usepackage{hyperref}
\usepackage{pdflscape} % dlia landscape orientation 
\hypersetup{colorlinks,citecolor=black,filecolor=black,linkcolor=black,urlcolor=black}
\usepackage{float}
\usepackage{setspace}
\usepackage{titlesec}
\titleformat{\chapter}
  {\Large\bfseries} % format
  {}                % label
  {0pt}             % sep
  {\huge}           % before-code

\usepackage{fancyhdr}
\pagestyle{fancy}

\fancyhead[LE,RO]{\slshape  \rightmark} %should be used with "twoside" in documentcalss. Delat headeri kak v knigah: vneshnije storoni sovpadajut drug s drugom. 
\fancyhead[LO,RE]{\slshape  \leftmark}
\fancyfoot[C]{\thepage}
\lhead{}
\rhead{SE31520 Assignment: Car Insurance System}

\renewcommand{\familydefault}{\sfdefault}
\setcounter{secnumdepth}{0} % sections are level 1
\renewcommand{\thesection}{}
\makeatother
\begin{document}
\title{SE31520 Assignment: Car Insurance System}
\author{Edgar Ivanov\\ edi@aber.ac.uk \\ Department of Computer Science, Aberystwyth University}
\date{\today}
\maketitle

\newpage
\thispagestyle{empty}
\mbox{}

\tableofcontents

\section{Introduction}
There was a task to implement prototype system which would allow to user request a quotation for a car insurance policy. As part of this system we needed to write two applications. One is underwriter which would represent insurance company and a broker which would collect quote premiums from the different insurance companies. However for this assignment task was simplified and broker needed to communicate just with one insurance underwriter, such approach affected some of my design decisions. For example, unique number which allows user to retrieve previous quotations is stored on the underwriter side, however in real world application it would be generated for the user and stored on the broker side and then linked to all underwriters. Broker had to be developed by web technology of our choice for which I have chosen PHP, underwriter had to be developed in ROR, store data in relational database management system and use ORM support. . This document will describe design of the broker and underwriter systems, testing strategy and includes self evaluation section.  

\section{Architecture of the underwriter}
Write a section on the architecture of the underwriter application and
rationale for decisions made. As part of this, produce a UML
diagram(s) that shows the architecture of your application. The design
diagram I used for the CSA application discussed in class might be a
useful starting point. I drew mine using Powerpoint, but feel free to use
another tool or even to draw neatly by hand!

Underwriter application was developed using Ruby on Rails. It is a RESTful web application using exclusively JSON for the representation of the content and data exchange. HTTP is used for the communication and supports all usual HTTP methods like GET, PUT, PATCH, POST, DELETE. At the beginning I tried to implement XML support for the representation exchange but faced some issues which I couldn't overcome. After a further reading about data formats like XML, JSON, YAML I choose to use JSON since it seemed to be lightweight, human-readable, easy to implement on the broker side, as well as CRUD operations with support of JSON baked in. Figure ~\ref{fig:DatabaseDesign} represents database design. Users table holds customer information like name, surname, DOB etc. Vehicles table holds information about the car: registration number, mileage, car value, it is linked to the main users table by the user\_id field and has one-to-one relationship since for each query it is allowed to enter details for one car only. Since there may be a few incidents that resulted in the claim I decided to store them in the separate "driver\_history" table, this table is linked back to the customer with the user\_id field. Addresses

\begin{figure}[H]
\centering
\centerline{\includegraphics[scale=0.198]{./DatabaseDesign}}
\caption{Database Design}
\label{fig:DatabaseDesign}
\end{figure}

\begin{figure}[H]
\centering
\centerline{\includegraphics[scale=0.25]{./ClassDesign}}
\caption{Class Design}
\label{fig:ClassDesign}
\end{figure}


\section{Architecture of the RESTful broker}
Write a section on the architecture of the RESTful broker client,
providing rationale for decisions made.
\section{Test strategy}
Write a section on your test strategy. IMPORTANT: Provide a
screencast of your underwriter application and RESTful broker client
working (some free screen-casting tools can be found online). This
must focus on the broker to underwriter interworking and be no longer
than five minutes long.
\section{Self-evaluation}
Write a self-evaluation section. Say what mark you should be awarded
and why. Say what you found hard or easy, and what was omitted and
why. Provide an analysis of your design and the appropriateness, or
otherwise, of the technologies used.

\bibliographystyle{ieeetr}
\bibliography{bibl}

\end{document}