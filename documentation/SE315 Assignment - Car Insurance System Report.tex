 \documentclass[10pt,a4paper,headinclude=true,twoside]{report}
\usepackage[latin1]{inputenc}
\usepackage[a4paper]{geometry}
\usepackage{a4wide}
\usepackage{amsmath}
\usepackage{amsfonts}
\usepackage{amssymb}
\usepackage{graphicx}
\usepackage{hyperref}
\usepackage{pdflscape} % dlia landscape orientation 
\hypersetup{colorlinks,citecolor=black,filecolor=black,linkcolor=black,urlcolor=black}
\usepackage{float}
\usepackage{setspace}
\usepackage{titlesec}
\titleformat{\chapter}
  {\Large\bfseries} % format
  {}                % label
  {0pt}             % sep
  {\huge}           % before-code

\usepackage{fancyhdr}
\pagestyle{fancy}

\usepackage{tabularx}

\fancyhead[LE,RO]{\slshape  \rightmark} %should be used with "twoside" in documentcalss. Delat headeri kak v knigah: vneshnije storoni sovpadajut drug s drugom. 
\fancyhead[LO,RE]{\slshape  \leftmark}
\fancyfoot[C]{\thepage}
\lhead{}
\rhead{SE31520 Assignment: Car Insurance System}

\renewcommand{\familydefault}{\sfdefault}
\setcounter{secnumdepth}{0} % sections are level 1
\renewcommand{\thesection}{}
\makeatother

\begin{document}
\title{SE31520 Assignment: Car Insurance System}
\author{Edgar Ivanov\\ edi@aber.ac.uk \\ Department of Computer Science, Aberystwyth University}
\date{\today}
\maketitle

\newpage
\thispagestyle{empty}
\mbox{}

\tableofcontents

\section{Introduction}
The assignment task was to implement the prototype system which would allow a customer to request a price for the car insurance. We were required to write two applications: the first one is so-called "underwriter" application which represents insurance company and the second one is broker application which would usually collect the quotes from the different insurance companies. For this assignment, however, the task was slightly simplified and broker needed to collect the quote from one insurance underwriter only.

I have developed "broker" and "underwriter" applications using different technologies: for the first one PHP was used and ROR was utilized to develop the second one. Broker application communicates with the underwriter using HTTP protocol and does it in a RESTful way. It is a great example of the interoperability, when applications can communicate despite the fact that they are written in different programming languages and are running on the different platforms. In the following I will describe the designs of the broker and underwriter systems, the testing strategy that was used, followed by the self evaluation section and the conclusions I have come up with. 

\section{Architecture of the underwriter}
%Write a section on the architecture of the underwriter application and
%rationale for decisions made. As part of this, produce a UML
%diagram(s) that shows the architecture of your application. The design
%diagram I used for the CSA application discussed in class might be a
%useful starting point. I drew mine using Powerpoint, but feel free to use
%another tool or even to draw neatly by hand!

The "Underwriter" application has been developed using Ruby on Rails and designed as a RESTful web service; it uses JSON for the representation of the content and data exchange. HTTP has been used for the communication and supports all the usual HTTP methods like GET, PUT, PATCH, POST, DELETE for the resource creation, deletion etc. In the beginning I tried to implement XML support for the representation exchange but faced some issues which I couldn't overcome. The further research conducted on the data formats like XML, JSON, YAML gave me the reasons to believe that JSON would be the best choice since it is lightweight, human-readable and it easy to implement JSON support on the broker side.

Figure ~\ref{fig:DatabaseDesign} shows the database design used to store the data about the customers. "Users" table holds the customer's information like name, surname, DOB etc. "Vehicles" table provides us with information about the car: the registration number, the mileage, the car value. It is linked to the main users table by the user\_id field and has one-to-one relationship. Since there may be a few incidents that would result in a claim I decided to store them in the separate "driver\_history" table; this table is linked to the customer with the user\_id field and holds the information about the incident's date,  the value claimed etc. It has one-to-many relationship with the user's table. "Addresses" table contains the information about the user's address: the street name, the postcode, the country etc.; it has one-to-one relationship since each user can only have one address registered in the system. "Quotations" table holds the quotes for each user and is linked to the user's table by user\_id field.    

\begin{figure}[H]
\centering
\centerline{\includegraphics[scale=0.198]{./DatabaseDesign}}
\caption{Database Design}
\label{fig:DatabaseDesign}
\end{figure}

On the figure ~\ref{fig:ClassDesign} there is a class diagram of the underwriter application. The idea is to use it as a business-to-business application; instead of generating HTML pages the interaction is done using the data formatted in JSON and is expected to be used by automated clients. There is nothing under the "view" in class diagram since it only responds to the data encoded in JSON format (HTML support was left for the debugging purpose to see what is held in the database, but it would be removed in production mode). To get a quote broker submits PUT request to the \textit{/users/new.json} address with the user data, then users controller handles this request and sends back the ID assigned to the newly created user encoded as JSON. I used the ID field in "Users" table as the unique reference number for the later quote retrieval. ID fields are handled by the rails and are guaranteed to be unique across all the users: it is one of the useful ways to identify a user in the future when the quote needs to be redisplayed. 

\begin{figure}[H]
\centering
\centerline{\includegraphics[scale=0.25]{./ClassDesign}}
\caption{Class Diagram}
\label{fig:ClassDesign}
\end{figure}



\section{Architecture of the RESTful broker}
The broker application was written as a web based application with the PHP support. It also uses cURL command line tool; it seems to be the only way to send JSON encoded data to the specific URL in the PHP. cURL is produced by the cURL computer software project and allows data to be transferred using various protocols \cite{cURL}. My server side scripting experience is limited to the use of SSI in HTML web pages. The fact that PHP is considered to be one of the most popular languages used for the server side coding \cite{PHP} gave me the reasons to believe that it was worth to try to use it. I had no experience in writing PHP code; this assignment was a great opportunity to make the first steps and learn to use it. The online tutorials have helped me to understand the basic concepts and start implementing my broker application.

The broker web application provides the opportunity for the potential customer to request a quote premium from the underwriter application (use case diagram is presented on the figure ~\ref{fig:usecase}). The customers interact with the broker only via HTTPS and the HTML formatted documents are transmitted to them. On the web page itself a customer needs to fill in and submit a form; on the next page a user is presented with the cost of the car insurance. This quote can be saved for the later retrieval; a customer is presented with the unique number which can be used to retrieve it. It also provides a customer with the link for a web page where the unique number can be entered and the quote will be redisplayed. The forms used on my web site were generated by the online form generator tool and modified accordingly to suite my needs. PHP is used to get the data from the forms and later on when building JSON objects containing customer data. With the help of cURL the broker application can use the standard HTTP methods to communicate with the underwrite application. To get a quotation the broker preforms the PUT request to the \textit{/users/new.json} address and receives ID number assigned to that user in the database. The broker then uses the ID received to retrieve the quotation from the underwriter using GET request and displays it to the user. On the figures ~\ref{fig:brokerMain}, ~\ref{fig:quotepremium}, ~\ref{fig:QuotationNumber}, ~\ref{fig:retrivequote} and ~\ref{fig:retrievedquote}. 


\begin{figure}[H]
\centering
\centerline{\includegraphics[scale=0.25]{./usecase}}
\caption{Use Case Diagram}
\label{fig:usecase}
\end{figure} 

\begin{figure}[H]
\centering
\centerline{\includegraphics[scale=0.45]{./brokerMain}}
\caption{Broker Home Page}
\label{fig:brokerMain}
\end{figure} 

\begin{figure}[H]
\centering
\centerline{\includegraphics[scale=0.45]{./quotepremium}}
\caption{Quote premium displayed}
\label{fig:quotepremium}
\end{figure} 

\begin{figure}[H]
\centering
\centerline{\includegraphics[scale=0.45]{./QuotationNumber}}
\caption{Quote Identifier}
\label{fig:QuotationNumber}
\end{figure}

\begin{figure}[H]
\centering
\centerline{\includegraphics[scale=0.45]{./retrivequote}}
\caption{Page to retrieve quote premium}
\label{fig:retrivequote}
\end{figure} 

\begin{figure}[H]
\centering
\centerline{\includegraphics[scale=0.45]{./retrievedquote}}
\caption{Retrieved quote premium}
\label{fig:retrievedquote}
\end{figure} 


\section{Test strategy}
Write a section on your test strategy. IMPORTANT: Provide a
screencast of your underwriter application and RESTful broker client
working (some free screen-casting tools can be found online). This
must focus on the broker to underwriter interworking and be no longer
than five minutes long.

\begin{center}
\begin{tabularx}{\textwidth} { |p{0.14cm}|p{0.88cm}|X|X|X|p{0.6cm}|X|}
   \hline                        
  ID &  Require ment & Description &  Inputs  &  Expected outputs & Pass/ Fail & Comments  \\ \hline
   1 & FR1 &  Request a quotation & Standard user details are entered &  Display page with quote premium & P &  \\ \hline
   2 & FR2 & Save a Quotation & "Get quote identifier" button is clicked &  Display page with unique identifier & P &  \\ \hline
   3 & FR3 & Retrieve a Saved Quotation & Identifier and user email are entered &  Redisplay page with quote premium & P & \\ \hline
   
   \hline  
\end{tabularx}
\end{center}
\section{Self-evaluation}
Write a self-evaluation section. Say what mark you should be awarded
and why. Say what you found hard or easy, and what was omitted and
why. Provide an analysis of your design and the appropriateness, or
otherwise, of the technologies used.

\bibliographystyle{ieeetr}
\bibliography{bibl}

\end{document}